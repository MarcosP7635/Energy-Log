\documentclass[12pt]{article}
\usepackage[utf8]{inputenc}
\usepackage{chemfig}
\usepackage[version=4]{mhchem}
\usepackage{amsfonts}
\usepackage{amsmath}
\usepackage{amssymb}
\usepackage{geometry}
\usepackage{mathabx}
\usepackage{relsize}
\usepackage{graphics}
\usepackage{outlines}
\usepackage[colorlinks = true,
            linkcolor = blue,
            urlcolor  = blue,
            citecolor = blue,
            anchorcolor = blue]{hyperref}
%\usepackage{indentfirst}
\usepackage{tikz}
\usepackage{listings}
\lstset{columns=fullflexible}
\usepackage{sidecap}
\usepackage{comment}
\geometry{
 a4paper,
 total={6.5in,0in},
 left= 15mm,
 top= 15mm,
 bottom=15mm,
 right = 15mm
 }
\title{Background Notes}
\author{Marcos Perez}
\date{January 2024 - }

\begin{document}

\maketitle

\section{Texts}
\begin{itemize}
    \item Direct Nuclear Reaction Theories by Norman Austern
    \item Griffith's Introduction to Quantum Mechanics 3rd Edition
\end{itemize}

\section{Wave Packets and Scattering}
\subsection{Warmup}
\subsubsection{Seperable Solutions to Time-Independent Schrodinger Equation}
\begin{align}
&\text{Recall the Schrodinger Equation (SE)}\quad i\hbar\partial_t\Psi=H\Psi\\
&\text{Consider solutions of the form}\quad \Psi(x,t)=\psi(x)\phi(t)\ \therefore\ \partial_t\Psi=\psi(x)\partial_t\phi(t)\ ,\ 
\partial^2_x\Psi=\phi(t)\frac{d^2\psi(x)}{dx^2}\\
\label{eqn:time_dep_factor}
&\text{With constant energy E}\quad \frac{E}{i\hbar}=\partial_t\Psi=\psi(x)\partial_t\phi(t)\quad\therefore\quad \phi(t)=e^{-i Et/\hbar}\\
\label{eqn:time_ind_SE}
&\text{With the momentum and position operators we also have}\quad -\frac{\hbar^2}{2m}\frac{d^2\psi(x)}{dx^2}+V(x)\psi(x)=E\psi(x) \\
&\text{And the general solution is of the form}\quad \Psi(x)=\sum_n\left[c_n\psi_n(x)e^{-itE_n/\hbar}\right]
\end{align}
$\phi(t)$ in Equation \ref{eqn:time_dep_factor} is known as the time-development operator and Equation \ref{eqn:time_ind_SE} is known as the time-independent SE.
\subsubsection{Free Particle}
\begin{align}
 &\text{Time-independent SE (Eq \ref{eqn:time_ind_SE}) and }V=0\quad -\frac{\hbar^2}{2m}\frac{d^2\psi(x)}{dx^2}=E\psi(x)\\
 &\text{Define }k\equiv\pm\sqrt{\frac{2mE}{\hbar}}\quad\therefore\quad \frac{d^2\psi}{dx^2}=-k^2\psi(x)\quad\text{Using Eq \ref{eqn:time_dep_factor}}\quad \mathlarger{\psi_k(x)=Ae^{ik\left(x-t\frac{k\hbar}{2m}\right)}}\\
 &\text{Notice that for discrete }k\quad \int_{-\infty}^\infty \Psi_k^*\Psi_kdx>c\ \forall\ c\in\mathbb{R}\\
&\text{Thus to be normalizable }k \text{ is continuous}\therefore\quad \Psi(x,t)=\left(2\pi\right)^{-1/2}\int_{-\infty}^\infty\phi(k)e^{ik\left(x-t\frac{k\hbar}{2m}\right)} dk
\\
&\text{Where by orthogonality}\quad \phi(k)=\left(2\pi\right)^{-1/2}\int_{-\infty}^\infty\psi(x,0)e^{ik\left(x-t\frac{k\hbar}{2m}\right)} dk
\end{align}
Such solutions with continuous indices are known as wave packets.
\section{5/31/22}







\end{document}
\documentclass[12pt]{article}
\usepackage[margin=1in]{geometry}
\usepackage{graphicx}
\usepackage{xcolor}
\usepackage{setspace}
\usepackage{titlesec}
\usepackage{float}
\usepackage [english]{babel}
\usepackage [autostyle, english = U.S.]{csquotes}
\MakeOuterQuote{"}
\titleformat{\subsubsection}
  {\color{gray}\normalfont\normalsize\selectfont}{\color{gray}\thesubsubsection}{1em}{}
\usepackage[hidelinks]{hyperref}
\usepackage{comment}
\setstretch{.9}
\usepackage[style=chem-acs,sorting=none]{biblatex}
\usepackage{titlesec}
\addbibresource{References.bib}
\geometry{
 a4paper,
 total={6.5in,0in},
 left= 15mm,
 top= 15mm,
 bottom=15mm,
 right = 15mm
 }
\begin{document}
\begin{center}
   \larger{\textbf{Mars-Tested Energy Storage (MTES): Safely Storing Energy With Nuclear Reactions} }\\
   Marcos Perez
\end{center}
\begin{comment}
    What do I want them to walk away with? \\
    Nuclear reactions 
\end{comment}
\textbf{Brief Overview}\par
Here I present a new battery technology in order to support a more sustainable society. Based on similar technology already used on the Mars rovers, these new batteries will store over 100 times as many Joules of energy per pound than current lithium-ion batteries. We seek to improve and replace an already ubiquitous technology using research replicated in several labs over the past 15 years. 
\\ \\ \textbf{Motivation}\par
Air pollution alone kills over 7 million people every year globally \cite{mannucci_novel_2019, noauthor_compendium_nodate, nansai_consumption_2021}. 
In order to reduce this death toll and combat climate change, it is imperative to improve energy storage technology to make solar and wind power more robust to the now frequent extreme changes in weather.
\\ \\ \textbf{Advantages Over Current Technology and Feasibility}\par
The batteries described below would have a specific energy 10$^{11}$ J/kg vs. the 10$^6$ J/kg of lithium-ion batteries. This 100-thousand-fold advantage is closer to a hundred-fold advantage when taking into account necessary shielding and the energy efficiency of current radioisotope thermal generators (RTGs). It is important to acknowledge that charging such a battery would be very expensive using current methodology. However, the sheer specific energy (J/kg) of such a technology warrants investigation into more energy efficient methods of energy storage with nuclear reactions. It is currently a small field in the preliminary stages of research, but in the long term it may hold great promise.
\\ \\ \textbf{Applications and Potential Market Share}\par 
Potential applications include emergency power supplies for hospitals and vehicles, any current application for lithium-ion batteries, and electric aircraft and spacecraft. Since most industries are incredibly reliant upon using large amounts of electricity, this technology has the potential to become incredibly commonplace.
\\ \\ \textbf{Battery Technology}\par
I seek to utilize recently discovered nuclear reactions in a safe and accessible manner to store energy that would be converted to electricity using the same technology that has operated continuously without fail for the past 12 years on the Mars rovers. Such a battery would operate using nuclear excitation which involves irradiating nuclei (the cores of atoms) in order to raise them to a stable excited energy level. By exciting these nuclei, they have essentially stored energy. By irradiating them a second time such that they are left in a higher yet much less stable
energy level, they will drop to their ground state and release their stored energy.  For nuclei that have potential for such an application, see my original comprehensive interactive plot: \href{https://marcosp7635.github.io/plots/Long_Lived_Isomers.html}{https://marcosp7635.github.io/plots/Long\_Lived\_Isomers.html}. The presentation would explain this plot at a level approachable people who are not in STEM. RTGs would then convert this released energy into electricity, and have successfully worked on the Mars Curiosity Rover since 2011 as well as the more recent Mars Perseverance Rover. I have a large library on Zotero of full-text PDFs and citations to be publicly accessible \href{https://www.zotero.org/groups/4913227/batteries_shared/collections/JRPIH93Y}{here: \\ https://www.zotero.org/groups/4913227/batteries\_shared/collections/JRPIH93Y}. I am not yet associated with a lab or an official company, but I am in the process of contacting nuclear reactor laboratories and particle accelerator groups to work on this technology as well as speaking to potential investors. 

\begin{comment}
 Originality, Innovativeness, Thoughtfulness
check
 Operational Feasibility and Sustainability
not feasible, incredibly sustainable
 Risk / Reward Advantage
high risk, the highest reward
 Positive Social / Market relevance and influence potential
wtf does this even mean
 Clarity and Organization
let's hope
\end{comment}   

\begingroup
\titleformat*{\section}{\fontsize{12pt}{14pt}\bfseries\selectfont}
\printbibliography
\endgroup


\end{document}